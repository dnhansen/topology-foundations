% Document setup
\documentclass[article, a4paper, 11pt, oneside]{memoir}
\usepackage[utf8]{inputenc}
\usepackage[T1]{fontenc}
\usepackage[UKenglish]{babel}

% Document info
\newcommand\doctitle{Conceptual foundations of topology}
\newcommand\docauthor{Danny Nygård Hansen}

% Formatting and layout
\usepackage[autostyle]{csquotes}
\usepackage[final]{microtype}
\usepackage{xcolor}
\frenchspacing
\usepackage{latex-sty/articlepagestyle}
\usepackage{latex-sty/articlesectionstyle}

% Fonts
\usepackage[largesmallcaps]{kpfonts}
\DeclareSymbolFontAlphabet{\mathrm}{operators} % https://tex.stackexchange.com/questions/40874/kpfonts-siunitx-and-math-alphabets
\linespread{1.06}
\let\mathfrak\undefined
\usepackage{eufrak}
\usepackage{inconsolata}
% \usepackage{amssymb}

% Hyperlinks
\usepackage{hyperref}
\definecolor{linkcolor}{HTML}{4f4fa3}
\hypersetup{%
	pdftitle=\doctitle,
	pdfauthor=\docauthor,
	colorlinks,
	linkcolor=linkcolor,
	citecolor=linkcolor,
	urlcolor=linkcolor,
	bookmarksnumbered=true
}

% Equation numbering
\numberwithin{equation}{chapter}

% Footnotes
\footmarkstyle{\textsuperscript{#1}\hspace{0.25em}}

% Mathematics
\usepackage{latex-sty/basicmathcommands}
\usepackage{latex-sty/framedtheorems}
\usepackage{latex-sty/topologycommands}
\usepackage{tikz-cd}
\tikzcdset{arrow style=math font} % https://tex.stackexchange.com/questions/300352/equalities-look-broken-with-tikz-cd-and-math-font
\usetikzlibrary{babel}

% Lists
\usepackage{enumitem}
\setenumerate[0]{label=\normalfont(\arabic*)}

% Bibliography
\usepackage[backend=biber, style=authoryear, maxcitenames=2, useprefix]{biblatex}
\addbibresource{references.bib}

% Title
\title{\doctitle}
\author{\docauthor}

\newcommand{\setF}{\mathbb{F}}
\newcommand{\ev}{\mathrm{ev}}
\newcommand{\calT}{\mathcal{T}}
\newcommand{\calU}{\mathcal{U}}
\newcommand{\calB}{\mathcal{B}}
\newcommand{\calE}{\mathcal{E}}
\newcommand{\calC}{\mathcal{C}}
\newcommand{\calD}{\mathcal{D}}
\newcommand{\calF}{\mathcal{F}}
\newcommand{\calG}{\mathcal{G}}
\newcommand{\calM}{\mathcal{M}}
\newcommand{\calA}{\mathcal{A}}
\newcommand{\calS}{\mathcal{S}}
\newcommand{\borel}{\mathcal{B}}
\newcommand{\measurable}{\mathcal{M}}
\newcommand{\wto}{\Rightarrow}
\DeclarePairedDelimiter{\net}{\langle}{\rangle}
\newcommand{\strucS}{\mathfrak{S}}
\DeclarePairedDelimiter{\gen}{\langle}{\rangle} % Generating set
\newcommand{\frakL}{\mathfrak{L}}

% Categories
\newcommand{\cat}[1]{\mathcal{#1}}
\newcommand{\scat}[1]{\mathbf{#1}} % category supposed to be small
\newcommand{\ncat}[1]{\mathbf{#1}} % named categories like Set, Top

\newcommand{\catSet}{\ncat{Set}} % Category of sets
\newcommand{\catGrp}{\ncat{Grp}} % Category of groups
\newcommand{\catTop}{\ncat{Top}} % Category of topological spaces
\newcommand{\catMble}{\ncat{Mble}} % Category of measurable spaces
\newcommand{\catPos}{\ncat{Pos}} % Category of posets
\newcommand{\catCLat}{\ncat{CLat}} % Category of complete lattices
\newcommand{\catCJoinLat}{\ncat{CsLat}^{\join}} % Category of complete join-semilattices
\newcommand{\catCMeetLat}{\ncat{CsLat}^{\meet}} % Category of complete meet-semilattices
\newcommand{\catStruc}[1]{\ncat{Str}_{#1}} % Category of structured sets
\newcommand{\catStrucS}{\catStruc{\strucS}} % Category of \strucS-structured sets

% TODO is this necessary?
% % Section style -- add to section style .sty?
% \setsubsecheadstyle{\normalfont\itshape}

\newcommand{\inpoint}{\prec}
\newcommand{\notinpoint}{\not\prec}
\newcommand{\calN}{\mathcal{N}}
\newcommand{\nhoods}[1]{\calN_{#1}}
\renewcommand{\powerset}[1]{\wp(#1)}

\newcommand{\pInt}[1]{\operatorname{pInt}(#1)}
\newcommand{\pCl}[1]{\operatorname{pCl}(#1)}
% \newcommand{\pCl}[1]{[#1]_p}
\renewcommand{\interior}[1]{\operatorname{Int}(#1)}
\renewcommand{\closure}[1]{\operatorname{Cl}(#1)}

\newenvironment{displaytheorem}{%
    \begin{displayquote}\itshape%
}{%
    \end{displayquote}%
}

% [TODO add to latex-sty?]
\renewcommand{\iff}{\Leftrightarrow}


\usepackage{adforn}

\begin{document}

\maketitle

\chapter{Pretopological spaces}

\section{Motivation and definitions}

Let $X$ be a set. Let $\inpoint$ be a relation from $X$ to $\powerset{X}$. If $x \inpoint U$, then we say that $x$ is an \emph{inner point} of $U$, and that $U$ is a \emph{neighbourhood} of $x$. One intuition for this relation is that $x \inpoint U$ just when $x$ is \enquote{completely contained} in $U$, or is \enquote{completely enclosed} in $U$, with \enquote{room to spare}. For this to be the case we obviously require that
%
\begin{enumerate}
    \item if $x \inpoint U$, then $x \in U$.
\end{enumerate}
%
Clearly the empty set cannot be a neighbourhood of any point. On the other hand, since $X$ is the entire universe under consideration we surely want $X$ to be a neighbourhood of $x$. Hence we have the postulates
%
\begin{enumerate}[resume]
    \item $x \notinpoint \emptyset$, and
    \item $x \inpoint X$.
\end{enumerate}
%
In particular, every point has at least one neighbourhood (namely $X$ itself). Furthermore, under the above interpretation, if $V \subseteq X$ is a set with $U \subseteq V$ then $V$ must also be a neighbourhood of $x$. Hence we have the postulate:
%
\begin{enumerate}[resume]
    \item If $x \inpoint U$ and $U \subseteq V$, then $x \inpoint V$.
\end{enumerate}
%
Next assume that both $U$ and $V$ are neighbourhoods of $x$. That is, $x$ is \enquote{completely contained} in both $U$ and $V$, so it seems natural enough that $x$ should be \enquote{completely contained} in the intersection $U \intersect V$. That is,
%
\begin{enumerate}[resume]
    \item If $x \inpoint U$ and $x \inpoint V$, then $x \inpoint U \intersect V$.
\end{enumerate}
%
% Finally, if $N$ is to be a neighbourhood of $x$, then a point $y \in X$ that is \enquote{close to $x$} must also be an inner point of $N$. If $N$ somehow \enquote{encloses} $x$, then it seems natural to require that the set of such $y$ also \enquote{enclose} $x$. We formalise this as a final postulate:
% %
% \begin{enumerate}[resume]
%     \item If $x \inpoint N$, then there is some $U \subseteq N$ such that $x \inpoint U$, and such that $y \inpoint N$ for all $y \in U$.
% \end{enumerate}

Now let $\nhoods{x}$ denote the collection of neighbourhoods of $x$. Then the above shows that $\nhoods{x}$ is a \emph{filter} on $X$, i.e. a proper, nonempty subset of $\powerset{X}$ that is upward closed and downward directed. Hence we call $\nhoods{x}$ the \emph{neighbourhood filter} of $x$. That is, a neighbourhood filter for a point $x \in X$ is a filter $\nhoods{x}$ on $X$ such that all sets in $\nhoods{x}$ contain $x$. A collection $(\nhoods{x})_{x \in X}$ of neighbourhood filters, one for each point in $X$, is then called a \emph{neighbourhood system} on $X$.

We summarise the discussion in the following definition:

\begin{definition}[Pretopological space]
    A \emph{pretopology} on a set $X$ is a relation $\inpoint$ from $X$ to $\powerset{X}$ with the following properties:
    %
    \begin{enumdef}
        \item Centred: For all $x \in X$ and $N \subseteq X$, if $x \inpoint N$ then $x \in N$.

        \item Left-total: For all $x \in X$ there exists an $N \subseteq X$ such that $x \inpoint N$.

        \item Downward directed: For all $x \in X$ and $N,M \subseteq X$, if $x \inpoint N$ and $x \inpoint M$ then $x \inpoint N \intersect M$.

        \item Upward closed: For all $x \in X$ and $N,M \subseteq X$, if $x \inpoint N$ and $N \subseteq M$ then $x \inpoint M$.
    \end{enumdef}
    %
    The pair $(X,\inpoint)$ is called a \emph{pretopological space}.

    If $x \inpoint N$ then $N$ is said to be a \emph{neighbourhood} of $x$, and $x$ an \emph{inner point} of $N$. The set
    %
    \begin{equation*}
        \nhoods{x} = \set{N \subseteq X}{x \inpoint N}
    \end{equation*}
    %
    is called the \emph{neighbourhood filter} of $x$, and the collection $(\nhoods{x})_{x \in X}$ is called the \emph{neighbourhood system} on $X$.
\end{definition}


\section{Closures and interiors}

Let $X$ be a pretopological space and consider distinct points $x,y \in X$. If there is an $N \in \nhoods{x}$ that does not contain $y$, then in some sense $x$ is \enquote{separated} from $y$, since it is \enquote{completely enclosed} in something that is disjoint from $y$. In this case we say that $x$ is \emph{separated from $y$}. Note however that even if $x$ is separated from $y$, $y$ might not be separated from $x$.

Next consider a subset $A \subseteq X$. If $a,a' \in A$ and $x \in X \setminus A$, then it may happen that $x$ is separated from $a$ but not from $a'$. It might also happen that $x$ is separated from every individual $a \in A$, so there is some $N_a \in \nhoods{x}$ with $a \not\in N_a$, but no matter which $N_a$ we pick there are still points $a' \in A$ that lie in $N_a$. However, if there is a $N \in \nhoods{x}$ that is completely disjoint from $A$, then we say that $x$ is \emph{separated from $A$}. On the other hand, a point that is \emph{not} separated from $A$ is in some sense \enquote{close} to $A$. Hence we have the following

\begin{definition}[Closure and preclosure]
    Let $X$ be a pretopological space, and let $A \subseteq X$.
    %
    \begin{enumdef}
        \item The \emph{preclosure} of $A$ is the set
        %
        \begin{equation*}
            \pCl{A}
                = \set{x \in X}{\forall N \in \nhoods{x} \colon A \intersect N \neq \emptyset},
        \end{equation*}
        %
        i.e. the set of points that are not separated from $A$.

        \item If $A = \pCl{A}$, then $A$ is called \emph{closed}.
        
        \item The \emph{closure} $\closure{A}$ of $A$ is the intersection of all closed sets that contain $A$.
    \end{enumdef}
\end{definition}

\begin{proposition}[Properties of preclosures]
    Let $X$ be a pretopological space, and let $A,B \subseteq X$. The preclosure operator $\pCl{\,\cdot\,}$ has the following properties:
    %
    \begin{enumprop}
        \item \label{enum:preclosure-increasing} Increasing: $A \subseteq B$ implies $\pCl{A} \subseteq \pCl{B}$.
        
        \item \label{enum:preclosure-extensive} Extensive: $A \subseteq \pCl{A}$.
        
        \item \label{enum:preclosure-nullary-unions} Preserves nullary unions: $\pCl{\emptyset} = \emptyset$
        
        \item \label{enum:preclosure-binary-unions} Preserves binary unions: $\pCl{A \union B} = \pCl{A} \union \pCl{B}$.
    \end{enumprop}
    %
    We also have the following:
    %
    \begin{enumprop}[resume]
        \item \label{enum:preclosure-characterisation-of-nhoods} A set $N \subseteq X$ is a neighbourhood of $x \in X$ if and only if $x \not\in \pCl{X \setminus N}$.

        \item \label{enum:preclosure-inclusion-series} There is a potentially infinite series of inclusions
        %
        \begin{equation*}
            A
                \subseteq \pCl{A}
                \subseteq \pCl{\pCl{A}}
                \subseteq \cdots
                \subseteq \closure{A}.
        \end{equation*}
        
        \item \label{enum:preclosure-equals-closure} If $A$ is closed, then $\pCl{A} = \closure{A}$.
    \end{enumprop}
\end{proposition}

\begin{proof}
    \begin{proofsec}
        \item[Proof of \subcref{enum:preclosure-increasing}]
        This is clear, since if a neighbourhood of $x \in X$ intersects $A$, then it also intersects $B$.
        
        \item[Proof of \subcref{enum:preclosure-extensive}]
        This is obvious, since if $x \in A$ then any neighbourhood of $x$ intersects $A$.
    
        \item[Proof of \subcref{enum:preclosure-nullary-unions}]
        This is obvious, since every neighbourhood has empty intersection with the empty set.
    
        \item[Proof of \subcref{enum:preclosure-binary-unions}]
        By \subcref{enum:preclosure-increasing} we have $\pCl{A} \subseteq \pCl{A \union B}$ and $\pCl{B} \subseteq \pCl{A \union B}$, so $\pCl{A} \union \pCl{B} \subseteq \pCl{A \union B}$. Conversely let $x \not\in \pCl{A} \union \pCl{B}$, so $x$ is both separated from $A$ and $B$. Hence there exist $N_1,N_2 \in \nhoods{x}$ such that $A \intersect N_1 = B \intersect N_2 = \emptyset$. But then $N_1 \intersect N_2$ is a neighbourhood of $x$ separating $x$ from $A \union B$. Hence $x \not\in \pCl{A \union B}$.
        
        \item[Proof of \subcref{enum:preclosure-characterisation-of-nhoods}]
        If $N$ is a neighbourhood of $x$, then $x$ has a neighbourhood disjoint from $X \setminus N$. Hence $x \not\in \pCl{X \setminus N}$. Conversely, if $x \not\in \pCl{X \setminus N}$ then $x$ has a neighbourhood $M$ disjoint from $X \setminus N$. But then we must have $M \subseteq N$, so $N$ is also a neighbourhood of $x$.

        \item[Proof of \subcref{enum:preclosure-inclusion-series}]
        Let $F \subseteq X$ be a closed set containing $A$. Then since $\pCl{F} = F$ we have $\pCl{A} \subseteq F$ by \subcref{enum:preclosure-increasing}. Repeatedly taking the preclosure on both sides we obtain
        %
        \begin{equation*}
            A
                \subseteq \pCl{A}
                \subseteq \pCl{\pCl{A}}
                \subseteq \cdots
                \subseteq F.
        \end{equation*}
        %
        Since $\closure{A}$ is intersection of all such $F$, the claim follows.

        \item[Proof of \subcref{enum:preclosure-equals-closure}]
        If $A$ is closed, then $A$ is one of the closed sets in the intersection defining $\closure{A}$, so $\closure{A} \subseteq A$. The claim then follows from \subcref{enum:preclosure-inclusion-series}.
    \end{proofsec}
\end{proof}

\begin{remark}
    A map $\pCl{\,\cdot\,} \colon \powerset{X} \to \powerset{X}$ satisfying properties \subcref{enum:preclosure-extensive}, \subcref{enum:preclosure-nullary-unions} and \subcref{enum:preclosure-binary-unions} is called a \emph{\v{C}ech closure operator} or a \emph{preclosure operator} on $X$. It is easy to see that these properties imply the other properties by mimicking the proofs above (note that we used \subcref{enum:preclosure-increasing} in the proof of \subcref{enum:preclosure-binary-unions} above). In fact, such an operator can also be used to define a pretopological structure on a set, by declaring that $N$ is a neighbourhood of $x$ if $x \not\in \pCl{X \setminus N}$, in accordance with property \subcref{enum:preclosure-characterisation-of-nhoods} above.
\end{remark}

Let us try to understand [TODO ref prop] the above results in terms of \enquote{closeness}:
%
\begin{proofsec}
    \item[Part \subcref{enum:preclosure-increasing}]
    Closeness is clearly increasing, in the sense that if $A \subseteq B$ and $x$ is close to $A$, then $x$ is also close to $B$.
    
    \item[Part \subcref{enum:preclosure-extensive}]
    Certainly every point is $A$ is close to $A$.

    \item[Part \subcref{enum:preclosure-nullary-unions}]
    Also, certainly nothing is close to the empty set.

    \item[Part \subcref{enum:preclosure-binary-unions}]
    If $x$ is close to either $A$ or $B$, then $x$ is certainly close to $A \union B$. The converse is less obvious: If $x$ is close to $A \union B$, is it really the case that $x$ is close to either $A$ or $B$? We might imagine a notion of closeness where closeness requires a certain point density in a particular area, and perhaps $A$ and $B$ have the required density when put together, but none of them are dense enough on their own.

    However, there might be a sense in which $x$ is not properly separated from either $A$ or $B$. Indeed, for $x$ to be separated from $A$, $x$ must have a neighbourhood $N$ disjoint from $A$. But that means that $x$ is \enquote{completely enclosed} in $N$, and that doesn't seem to be the case if $A$ has some density in the relevant area.
    
    \item[Part \subcref{enum:preclosure-characterisation-of-nhoods}]
    This is essentially just a restatement of the definition of preclosure. If $x$ is completely contained in $N$, then it is not close to $X \setminus N$.

    \item[Part \subcref{enum:preclosure-inclusion-series}]
    In general, the preclosure of a set is not closed. That is, the preclosure operator is not idempotent. In \enquote{closeness} terms, this means that points that are not close to $A$ might still be close to $\pCl{A}$. For example, say that an integer $n$ is close to a set $A \subseteq \ints$ if the smallest difference between $n$ and any $a \in A$ is at most $1$. Then
    %
    \begin{equation*}
        \pCl{A}
            = \bigunion_{a \in A} \{a-1, a, a+1\},
    \end{equation*}
    %
    which is clearly not idempotent, though it is a \v{C}ech closure operator.

    If the sequence of inclusions in \subcref{enum:preclosure-inclusion-series} terminates, then we have found all the points that are close to $A$, and the resulting set is closed. A set is thus closed if it contains all points that are close to it. But for the preclosure on $\ints$ above, the only closed sets are $\emptyset$ and $\ints$ itself.

    % In [TODO ref] we will return to this example. We note finally that a way to assign a neighbourhood system to $\ints$ that gives rise to the above preclosure operation is to let
    % %
    % \begin{equation*}
    %     \nhoods{x}
    %         = \set{N \subseteq \ints}{x-1,x,x+1 \in N}
    % \end{equation*}
    % %
    % for all $x \in \ints$.
\end{proofsec}

A second important operation is the following:

\begin{definition}[Interior and preinterior]
    Let $(X,\inpoint)$ be a pretopological space, and let $A \subseteq X$.
    %
    \begin{enumdef}
        \item The \emph{preinterior} of $A$ is the set
        %
        \begin{equation*}
            \pInt{A}
                = \set{x \in A}{x \inpoint A}
        \end{equation*}
        %
        of all inner points of $A$.

        \item If $A = \pInt{A}$, then $A$ is called \emph{open}.
        
        \item The \emph{interior} $\interior{A}$ of $A$ is the union of all open sets contained in $A$.
    \end{enumdef}
\end{definition}

\begin{remark}
    In a topological space, if $N$ is a neighbourhood of a point $x$, then there exists an open set $U$ such that $x \in U \subseteq N$. We claim that this is not necessarily the case in a pretopological space: Indeed, we show that the interior of a neighbourhood can be empty.

    Consider the pretopological space whose underlying set is $\reals$, and where $\nhoods{0} = \set{N \subseteq \reals}{(-1,1) \subseteq N}$ and $\nhoods{x} = \{\reals\}$ for $x \neq 0$. Then $(-1,1)$ is a neighbourhood of $0$, but if $A \subseteq (-1,1)$ is to be open, then we must have $\pInt{A} = A$, and this is only possible when $A = \emptyset$.

    The trouble is clearly that $(-1,1)$ is a neighbourhood of $0$ but not of any of the points that are close to $0$, which it would be in a topological space.
\end{remark}


Closures and interiors are dual to each other:
%
\begin{proposition}
    Let $(X,\inpoint)$ be a pretopological space, and let $A \subseteq X$. Then
    %
    \begin{equation*}
        X \setminus \pCl{A} = \pInt{X \setminus A}
        \quad \text{and} \quad
        \pCl{X \setminus A} = X \setminus \pInt{A}.
    \end{equation*}
    %
    In particular, a set is open if and only if its complement is closed, and
    %
    \begin{equation*}
        X \setminus \closure{A} = \interior{X \setminus A}
        \quad \text{and} \quad
        \closure{X \setminus A} = X \setminus \interior{A}.
    \end{equation*}
\end{proposition}

\begin{proof}
    The first claim follows from the following series of equivalences:
    %
    \begin{align*}
        x \not\in \pCl{A}
            &\;\iff\; \exists N \in \nhoods{x} \colon A \intersect N = \emptyset \\
            &\;\iff\; \exists N \in \nhoods{x} \colon N \subseteq X \setminus A \\
            &\;\iff\; X \setminus A \in \nhoods{x} \\
            &\;\iff\; x \in \pInt{X \setminus A}.
    \end{align*}
    %
    The second part follows by replacing $A$ by $X \setminus A$ and taking complements on both sides.
    
    If $F \subseteq X$ is closed, then
    %
    \begin{equation*}
        X \setminus F
            = X \setminus \pCl{F}
            = \pInt{X \setminus F}
            \subseteq X \setminus F,
    \end{equation*}
    %
    so $X \setminus F$ is open. Conversely, if $U \subseteq X$ is open then
    %
    \begin{equation*}
        X \setminus U
            = X \setminus \pInt{U}
            = \pCl{X \setminus U}
            \supseteq X \setminus U,
    \end{equation*}
    %
    which implies that $\pCl{X \setminus U} = X \setminus U$, so $X \setminus U$ is closed.
    
    Finally, let $\calF$ be the collection of closed sets containing $A$, and let $\calU$ be the collection of open sets contained in $X \setminus A$. Then the duality between open and closed sets implies that
    %
    \begin{equation*}
        X \setminus \closure{A}
            = X \setminus \bigintersect_{F \in \calF} F
            = \bigunion_{F \in \calF} X \setminus F
            = \bigunion_{U \in \calU} U
            = \interior{X \setminus A},
    \end{equation*}
    %
    as claimed. The second part of the last claim follows by taking complements.
\end{proof}

We may use this correspondence to obtain properties of preinteriors analogous to those of preclosures in [TODO ref], but we will not use these in the sequel.




\section{Convergence and continuity}

We begin by considering continuity. Intuitively, a function $f \colon X \to Y$ is continuous at a point $x \in X$ if points in $X$ that are close to $x$ are mapped to points in $Y$ that are close to $f(x)$. Usually this is formalised as follows:

\begin{definition}[Continuity]
    Let $X$ and $Y$ be pretopological spaces, and let $x \in X$. A map $f \colon X \to Y$ is \emph{continuous at $x$} if, for all $A \subseteq X$, $x \in \pCl{A}$ implies that $f(x) \in \pCl{f(A)}$, i.e.
    %
    \begin{equation*}
        \forall A \subseteq X \colon x \in \pCl{A} \implies f(x) \in \pCl{f(A)}.
    \end{equation*}
    %
    If $f$ is continuous at $x$ for all $x \in X$, then we say that $f$ is \emph{continuous}.
\end{definition}
%
That is, we require that if $A$ is a set that $x$ is close to, then $f(A)$ is a set that $f(x)$ is close to. But this seems like the opposite of what we were saying before, since there is a difference between points close to $x$ and points that $x$ is close to. This is obvious in our formalism, since the \enquote{close to} relation is heterogeneous: Indeed, points can only be close to \emph{sets}, not to points, nor can sets be close to sets. Hence if $x$ is supposed to be close to something, or something is supposed to be close to $x$, then that something must be a set.

Indeed, it seems difficult to define a general \emph{homogeneous} closeness relation using closures, since if $x \in A$ and $y \in B$, then $x$ might lie in $\pCl{B}$ while $y$ doesn't lie in $\pCl{A}$. Hence we use the heterogeneous relation above.

As in topological spaces we can characterise global continuity more simply:

\begin{proposition}
    Let $f \colon X \to Y$ be a map between pretopological spaces. Then $f$ is continuous if and only if
    %
    \begin{equation*}
        f(\pCl{A})
            \subseteq \pCl{f(A)}
    \end{equation*}
    %
    for all $A \subseteq X$.
\end{proposition}

\begin{proof}
    If $f$ is continuous and $A \subseteq X$, then for all $x \in \pCl{A}$ we have $f(x) \in \pCl{f(A)}$, which just says that $f(\pCl{A}) \subseteq \pCl{f(A)}$. Conversely, if $f$ satisfies the above condition and $x \in \pCl{A}$, then $f(x) \in f(\pCl{A}) \subseteq \pCl{f(A)}$, so $f$ is continuous at $x$.
\end{proof}

Next we wish to explicitly relate continuity to neighbourhoods, as is standardly done in topology. In a topological space it is well-known that continuity as defined above is equivalent to preimages of open sets being open. In pretopological spaces we have something similar, though we must focus on neighbourhoods:

\begin{proposition}
    Let $f \colon X \to Y$ be a map between pretopological spaces, and let $x \in X$. Then $f$ is continuous at $x$ if and only if it has the following property: For all $N \in \nhoods{f(x)}$ there is an $M \in \nhoods{x}$ such that $f(M) \subseteq N$, i.e.
    %
    \begin{equation*}
        \forall N \in \nhoods{f(x)} \exists M \in \nhoods{x} \colon f(M) \subseteq N.
    \end{equation*}
    %
    In other words, $f$ is continuous at $x$ if $f\preim(N) \in \nhoods{x}$ for all $N \in \nhoods{f(x)}$.
\end{proposition}

\begin{proof}
    First assume that $f$ has the above property. Let $A \subseteq X$ be such that $x \in \pCl{A}$, and let $N \in \nhoods{f(x)}$. Then there exists an $M \in \nhoods{x}$ such that $f(M) \subseteq N$. Since $M$ is a neighbourhood of $x$ we have $A \intersect M \neq \emptyset$. It follows that
    %
    \begin{equation*}
        \emptyset
            \neq f(A \intersect M)
            \subseteq f(A) \intersect f(M)
            \subseteq f(A) \intersect N,
    \end{equation*}
    %
    so since $N$ was arbitrary we have $f(x) \in \pCl{f(A)}$ as desired.

    Conversely, assume that $f$ is continuous at $x$, and let $N \in \nhoods{f(x)}$. Assume towards a contradiction that $x \in \pCl{f\preim(Y \setminus N)}$. Since $f(f\preim(Y \setminus N)) \subseteq Y \setminus N$ we then have, by continuity,
    %
    \begin{equation*}
        f(x)
            \in \pCl{Y \setminus N}
            = Y \setminus \pInt{N},
    \end{equation*}
    %
    which is a contradiction since $f(x) \in \pInt{N}$. Hence $x$ has a neighbourhood $M$ disjoint from $f\preim(Y \setminus N)$, and this has the property that $f(M) \subseteq N$.
\end{proof}

Let us try to understand this result: If $f \colon X \to Y$ is continuous at $x \in X$, then our initial definition is supposed to say that $f$ takes points in $X$ that are close to $x$ and map them to points in $Y$ that are close to $f(x)$. What the above says is that if (and only if) $f$ is continuous, then no matter which neighbourhood $N$ of $f(x)$ we consider, we can always find a neighbourhood $M$ of $x$ such that $f$ maps $M$ to $N$. Since it is in some sense \enquote{harder} to map a set into $N$ the smaller $N$ is -- i.e., we might have to require $M$ to be smaller if $N$ is made smaller -- this seems to suggest another interpretation of neighbourhoods: Namely as formalising a sense of closeness. It is sometimes put as follows:

Every neighbourhood $N$ of a point $x \in X$ defines a notion of closeness, let's call it \emph{$N$-closeness}. We say that a point $x' \in X$ is $N$-close to $x$ if $x' \in N$. We may then restate [TODO ref] as follows:
%
\begin{displaytheorem}
    A map $f \colon X \to Y$ is continuous at $x \in X$ if and only if it has the following property: For every $N \in \nhoods{f(x)}$ there is an $M \in \nhoods{x}$ such that if $x'$ is $M$-close to $x$, then $f(x')$ is $N$-close to $f(x)$.
\end{displaytheorem}
%
This is then entirely analogous to the $\epsilon$-$\delta$-definition of continuity in metric spaces, where we might have said that $x'$ is $\delta$-close to $x$ if $\rho(x,x') < \delta$, though of course $N$-closeness is not generally symmetric.

We might wonder whether the above gives us the right notion of closeness. What would happen, for instance, if we for $x'$ to be $N$-close to $x$ required that $x'$ be an \emph{inner point} of $N$? In a topological space we might think of this as more natural, especially if we are accustomed to thinking of neighbourhoods as by definition open. This would also not \emph{prima facie} contradict our earlier interpretation of neighbourhoods as \enquote{wholly containing} its inner points. I do not know of any conceptual problems with this approach, but while the two are practically equivalent in topological spaces, in a \emph{pretopological} space neighbourhoods of a point do not behave nicely with respect to points that are \enquote{close} to it, as we saw in [TODO ref rem].

\newcommand\fleuronbreak{\fancybreak{\textcolor{linkcolor}{\adfhangingflatleafleft}}}

\fleuronbreak

We now turn to convergence. Let $\calF$ be a system of subsets of a pretopological space $X$. We think of the sets in $\calF$ as describing the state of some system $\calS$ in $X$, where $F \in \calF$ means that $\calS$ eventually lies in $F$. If $F \subseteq F'$ then $\calS$ of course also eventually lies in $F'$. And for any two $F_1, F_2 \in \calF$, if $\calS$ eventually lies in $F_1$ and eventually lies in $F_2$, then it seems natural that $\calS$ will eventually lie in $F_1 \intersect F_2$. Finally, clearly the system cannot lie in the empty set, and on the other hand it trivially lies in $X$, so in total $\calF$ is a filter on $X$.

Let us consider what it can mean for $\calS$ to \enquote{converge} to a point $x \in X$. Can we use preclosures to define such a notion? This seems rather difficult, since $\calS$ is not a set so it does not have a preclosure. All we know is that $\calS$ somehow \enquote{eventually} lies in $F$ for all $F \in \calF$,\footnote{This kind of situation is often described as $\calS$ lying in all $F \in \calF$, but this wording makes it easy to conflate the current situation with $\calS$ lying in all $F$ \emph{simultaneously}.} but the exact nature of $\calS$ is unknown to us. Perhaps we can instead use the notion of closeness given to us by the neighbourhood filter $\nhoods{x}$ of $x$: If we trust this notion, then for $\calS$ to converge to $x$ we might require that, for all $N \in \nhoods{x}$, $\calS$ is eventually $N$-close to $x$. That is, $\calS$ should eventually lie in $N$. But since $\calF$ is precisely the collection of sets in which $\calS$ eventually lies, this just means that we should require that $\nhoods{x} \subseteq \calF$. This naturally leads to the following definition:

\begin{definition}[Convergence]
    Let $X$ be a pretopological space, and let $x \in X$. A filter $\calF$ on $X$ \emph{converges} to $x$ if $\nhoods{x} \subseteq \calF$.

    Equivalently, a net $u$ in $X$ converges to $x$ if $u$ is eventually in $N$ for all $N \in \nhoods{x}$.
\end{definition}
%
That is, convergence of filters and nets are entirely analogous to convergence in topological spaces.

Before presenting the next result, recall that the \emph{pushforward} of a filter $\calF$ on a set $X$ by a set function $f \colon X \to Y$ is the filter $f(\calF)$ on $Y$ generated by the filter basis $\set{f(F)}{F \in \calF}$. Let us try to understand this construction conceptually: If $\calS$ is some system in $X$, then we might imagine mapping that to $Y$ by $f$ to obtain a system $f(\calS)$. For instance, if $\calS$ is a net $u$, then $f(\calS)$ is just the net $f \circ u$. If $\calS$ eventually lies in some set $F \subseteq X$, then $f(\calS)$ will simultaneously lie in $f(F)$. If $\calF$ is the filter of sets in which $\calS$ eventually lie, then $f(\calS)$ will thus eventually lie in all sets in $\set{f(F)}{F \in \calF}$, and hence in all the sets contained in the filter that this collection (which is indeed a filter basis) generates.

Next suppose that $\calF$ converges to some $x \in X$. What about the limit behaviour of the pushforward $f(\calF)$? Does this converge to $f(x)$? For it to do so we require that $\nhoods{f(x)} \subseteq f(\calF)$, i.e. $f(\calF)$ should eventually lie in $N$ for all neighbourhoods $N$ of $f(x)$. When are we sure of this if we already know that $\calF$ eventually lies in $M$ for all neighbourhoods $M$ of $x$? If $f$ is continuous then this should certainly happen: For then if $\calS$ just gets close enough to $x$, then $f(\calS)$ can get as close as desired to $f(x)$. More precisely we have the following result:

\begin{proposition}[Continuity, filters and nets]
    \label{thm:continuity-filter-net}
    Let $f \colon X \to Y$ be a function between pretopological spaces, and let $x \in X$. Then $f$ is continuous at $x$ if and only if $\nhoods{f(x)} \subseteq f(\nhoods{x})$, i.e. if $f(\nhoods{x}) \to f(x)$. In particular, the following are equivalent:
    %
    \begin{enumprop}
        \item \label{enum:continuity-point} $f$ is continuous at $x$.
        \item \label{enum:filter-convergence-point} For all filters $\calF$ on $X$, $\calF \to x$ implies that $f(\calF) \to f(x)$.
        \item \label{enum:net-convergence-point} For all nets $u$ in $X$, $u \to x$ implies that $f(u) \to f(x)$.
    \end{enumprop}
\end{proposition}

\begin{proof}
    If $f$ is continuous at $x$ and $N \in \nhoods{f(x)}$, then there exists an $M \in \nhoods{x}$ such that $f(M) \subseteq N$. But $f(M)$ lies in $f(\nhoods{x})$, and hence so does $N$.

    Conversely, if $\nhoods{f(x)} \subseteq f(\nhoods{x})$ and $N \in \nhoods{f(x)}$, then by definition of $f(\nhoods{x})$ there exists an $M \in \nhoods{x}$ such that $f(M) \subseteq N$, so $f$ is continuous at $x$.
%
\begin{proofsec}
    \item[\subcref{enum:continuity-point} $\implies$ \subcref{enum:filter-convergence-point}]
    Recall that $\calF \to x$ means that $\nhoods{x} \subseteq \calF$. Hence $\nhoods{f(x)} \subseteq f(\nhoods{x}) \subseteq f(\calF)$, so $f(\calF) \to f(x)$.

    \item[\subcref{enum:filter-convergence-point} $\implies$ \subcref{enum:continuity-point}]
    This follows by setting $\calF = \nhoods{x}$.

    \item[\subcref{enum:filter-convergence-point} $\Leftrightarrow$ \subcref{enum:net-convergence-point}]
    This follows immediately from the correspondence between nets and filters.
\end{proofsec}
\end{proof}
%
The first part of the proposition implies that
%
\begin{equation*}
    \nhoods{x} \to x
    \quad \implies \quad
    f(\nhoods{x}) \to f(x)
\end{equation*}
%
whenever $f$ is continuous at $x$ (the antecedent is of course always true, so this implication is true just when the consequent is). The content of the implication \subcref{enum:continuity-point} $\implies$ \subcref{enum:filter-convergence-point} above is that we may replace $\nhoods{x}$ by any filter whatsoever.


\nocite{*}

\printbibliography


\end{document}
